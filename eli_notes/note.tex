\documentclass[10pt, oneside]{article}
\usepackage{mathtools, amsmath, amsthm, amssymb, calrsfs, wasysym, verbatim, bbm, color, graphics, geometry, siunitx, tikz, dsfont, setspace}

\doublespacing

\geometry{tmargin=.75in, bmargin=.75in, lmargin=.75in, rmargin = .75in}


\def\checkmark{\tikz\fill[scale=0.4](0,.35) -- (.25,0) -- (1,.7) -- (.25,.15) -- cycle;} 

\newcommand{\R}{\mathbb{R}}
\newcommand{\C}{\mathbb{C}}
\newcommand{\Z}{\mathbb{Z}}
\newcommand{\N}{\mathbb{N}}
\newcommand{\Q}{\mathbb{Q}}
\newcommand{\Cdot}{\boldsymbol{\cdot}}

\DeclarePairedDelimiter\bra{\langle}{\rvert}
\DeclarePairedDelimiter\ket{\lvert}{\rangle}
\DeclarePairedDelimiterX\braket[2]{\langle}{\rangle}{#1\,\delimsize\vert\,\mathopen{}#2}

\newtheorem{thm}{Theorem}
\newtheorem{defn}{Definition}
\newtheorem{post}{Postulate}
\newtheorem{qs}{Question}
\newtheorem{lem}{Lemma}
\newtheorem{cor}{Corollary}


\title{Emerging \& Low Dimensional Materials: [PHYS-789]}
\author{M. Elijah Wangeman}
\date{Spring 2024}

\usepackage{color}   %May be necessary if you want to color links
\usepackage[hidelinks]{hyperref}
\hypersetup{
    colorlinks=false, %set true if you want colored links
    linktoc=all,     %set to all if you want both sections and subsections linked
}

% TIKZ
\usetikzlibrary{arrows} 

\newcommand{\sga}[6]
{
    % analyzer body
    \draw[rounded corners] (#3-1, #4-0.5) rectangle (#3+1, #4+0.5) {};
    % analyzer labels
    \draw (#3, #4+0.75) node[font=\small] {#2};
    \draw (#3, #4) node {#1};
    % output high
    \draw (#3+0.75, #4+0.2) node {$+$};
    \draw [thick, ->, >=stealth'](#3+1,#4+0.2) -- (#3+2,#4+0.2);
    \draw (#3+2.4, #4+0.2) node {$#5$};
    % output low
    \draw (#3+0.75, #4-0.2) node {$-$};
    \draw [thick, ->, >=stealth'](#3+1,#4-0.2) -- (#3+2,#4-0.2);
    \draw (#3+2.4, #4-0.2) node {$#6$};
}

\newcommand{\sgal}[4]
{
    \draw[rounded corners] (#3-1, #4-0.5) rectangle (#3+1, #4+0.5) {};
    \draw (#3, #4+0.75) node[font=\small] {#2};
    \draw (#3, #4) node {#1};
    \draw (#3-0.75, #4+0.2) node {$+$};
    \draw [thick, ->, >=stealth'](#3-1,#4+0.2) -- (#3-2,#4+0.2);
    \draw (#3-0.75, #4-0.2) node {$-$};
    \draw [thick, ->, >=stealth'](#3-1,#4-0.2) -- (#3-2,#4-0.2);
}

\newcommand{\srcp}[2]
{
    \draw[rounded corners, fill=lightblue] (#1-1, #2-0.5) rectangle (#1+1, #2+0.5) {};
    \draw (#1, #2) node[font=\small] {Pair Source};
    \draw [thick, ->, >=stealth'](#1-1,#2) -- (#1-2,#2);
    \draw [thick, ->, >=stealth'](#1+1,#2) -- (#1+2,#2);
}

\def\checkmark{\tikz\fill[scale=0.4, color=lightgreen](0,.35) -- (.25,0) -- (1,.7) -- (.25,.15) -- cycle;} 
\newcommand\xmark{%
  \tikz[scale=0.4,red]{
    \fill(0,0)--(0.1,0) .. controls (0.5,0.4) .. (1,0.7)--(0.9,0.7) ..  controls (0.5,0.5) ..(0,0.1) --cycle;
    \fill(1,0.1)--(0.9,0.1) .. controls (0.5,0.3) .. (0,0.7)--(0.1,0.7) .. controls (0.5,0.4) ..(1,0.2) --cycle;
  }%
}

\begin{document}

\maketitle
\tableofcontents

\vspace{.25in}


\section{Wednesday, January 17: }

\subsubsection{Doping, Transistor Characteristics \& FETs}
\begin{itemize}
\item Doping: adding impurities to a semiconductor to change its electrical properties.\\
    - n-type: add electrons (donor)\\
    - p-type: add holes (acceptor)
\item Transistor: a semiconductor device used to amplify or switch electronic signals and electrical power.\\
    - BJT: bipolar junction transistor\\
    - FET: field-effect transistor
\item FET: a transistor in which current flows through a semiconductor channel whose width is modulated by an electric field.\\
    - MOSFET: metal-oxide-semiconductor field-effect transistor
    - Family of curves: $I_D$ vs $V_{DS}$ for different values of $V_{GS}$
    - Saturation: $V_{DS} > V_{GS} - V_{TH}$

\end{itemize}

\subsubsection{Logic Gates}
\begin{itemize}
\item Inverter: $V_{out} =  \text{inverse of } V_{in}$
\end{itemize}

\subsubsection{What are 2D \& low dimensional materials?}

Crystalline single layer materials with thickness of a few atoms.\\
0D, 1D, 2D, 3D materials.
\begin{itemize}
\item 0D: 
\item 1D: Carbon nanotubes, nanowires.
\item 2D: Graphene, transition metal dichalcogenides (TMDs), black phosphorus, hexagonal boron nitride (hBN).
\item 3D: Bulk materials.
\end{itemize}


























\end{document}
